\documentclass[a4paper,10pt]{article}
\usepackage[utf8]{inputenc}

%opening
\title{4º Trabalho de Compiladores - Interpretador para a linguagem Humildade++}
\author{André Pacheco e Thaylo Xavier}

\begin{document}

\maketitle

\begin{abstract}
Documentação referente à quarta parte do trabalho da disciplina de compiladores: Implementação de um interpretador para a linguagem Humildade++ baseado na linguagem G-Portugol.
\end{abstract}

\section{Introdução}
O objetivo deste trabalho é a construção de um interpretador para a linguagem Humildade++, que é baseada na linguagem G-Portugol.
Para o desenvolvimento do interpretador, é necessário a utilização dos três analisadores desenvolvidos nos trabalhos anteriores, ou seja,
os trabalhos anteriores são incorporados ao interpretador, criado aqui, fazendo com que o interpretador verifique erros léxicos, sintáticos e
semânticos, antes de executar as instruções da linguagem. A seguir serão descritas as principais ideias, linhas de códigos e dificuldades
encontradas, para o desenvolvimento do interpretador.


\section{Menu}
	O interpertador H++ disponibiliza um menu para o usuário escolher o que deve ser feito. O menu consiste das seguintes opções:

\begin{enumerate}
 \item [1] Compilar programa
 \item [2] Executar programa
 \item [3] Listar programas no diretório corrente
 \item [4] Listar programas compilados
 \item [5] Imprimir árvore de execução
 \item [6] Sair 
\end{enumerate}

	Portanto, as opções descritas no menu são basicamente o que deve ser implementado aqui no interpretador.	

\section{Visão geral}
	Como já foi dito, o interpretador utiliza os analisadores já feitos anteriormente, portanto, as estruturas de dados utilizadas no analisador 
	sintático, são "herdadas" aqui. As principais estruturas criadas no interpretador são a MasterNode e a PilhaArv. Ambas são descritas a seguir:
	\begin{verbatim}
     struct MasterNode
     {
        int id_unico;
        char str1[50];
        char str2[50];
        void *data; 
        int dataType;
        Variavel *var;
        Funcao *func; 
        struct MasterNode *prox;
        struct MasterNode *filhos[3];
     };		
	\end{verbatim}	
	\begin{verbatim}
    struct PilhaArv
    {
       MasterNode *pilha[max_stack_size_t];
       int quant;
    };
	\end{verbatim}	
	A estrutura MasterNode é uma lista e árvore ao mesmo tempo. Ela pode ter no máximo três filhos, na esquerda, no meio e na direita,
	caracterizando uma árvore, e ela também aponta para próxima estrutura de mesma forma. Nela são contruída a árvore de cada comando do programa e
	encadeado todas as instruções encontradas nele. Além disso, ela possui um id único, que é utilizado como debuger, duas strings, para indicar
	o que é a instrução lida, uma ponteiro para void, para armazenar qualquer valor constante, um inteiro que representa o tipo do dado armazenado,
	e um ponteiro para a estrutura Variável e outro para estrura Funcao, ambas já descritas no analisador semântico.\\
	
	Para armazenar todas as instruções existe uma dificuldade para saber se um comando é "pai" de algum outro, se ele é "filho" ou se ele está
	sozinho. Para isso foi criado a estrutura PilhaArv, que nada mais é que uma pilha de MasterNode. A ideia é seguinte: Quado encontramos um
	comando da linguagem, já sabemos de antemão que ele possui um número pré-determinado de filhos. Portanto, quando instruções são encontradas
	elas necessáriamente devem ser empilhadas até que se encontre o "pai" delas. E como já é conhecido o número de filhos de cada "pai", ele 
	desempilha exatamente esse número, tendo como "filhos", o número de intruções desempilhados. Utilizando um exemplo para o que foi dito acima
	temos:\\
	Utilizando como exemplo uma expressão \textit {x := x + 1}. Inicialmente empilha-se 1 e x e quando encontra-se o operador +, sabemos que deve
	ser dempilhados duas posições da pilha corrente, ou seja, 1 e x, que se tornam filhos do operador +. Feito isso, o nó do operador + deve ser
	empilhado. Em seguida, empilha-se o x mais a esquerda e encontra-se o operador de atribuição. Quando é encontrado este, sabemos que deve ser
	desempilhado duas posições da pilha, ou seja, a expressão de soma + e x, que se tornam filhos do operador de atribuição, que ao final de 
	tudo isso, se empilha, pois pode pertencer a um bloco de comando, como um um se-então, por exemplo, ou simplismente fica na pilha e ao fim
	da anáise do programa, será um nó na lista de masterNode apontando para um próximo nó.
	\\
	
\section{Criação da Pilha e MasterNode}
	A criação da pilha e do MasterNode andam lado a lado. Para implementar toda ideia descrita na seção anterior, foi criado os arquivos 
	\textit Expressao.c e \textit Comandos.c; Sendo assim, separou-se as expressões de comando para construção de ambas as estruturas.
	Nestes arquivos pode ser encontrado todos os passos para cada tipo de comando. A seguir é exibido um trecho de código para tratamento de
	um comando:
	
\begin{verbatim}
    case cmd_se: 
    {
        MasterNode *m = criaNode("cmd", "se");
        insereNode(m,desempilhaNode(pArv),'d');	// ELSE
        insereNode(m,desempilhaNode(pArv),'m'); // THEN
        insereNode(m,desempilhaNode(pArv),'e'); // EXP
        empilhaNode(pArv,m); // se empilhando
        break;
    }
\end{verbatim}

	Acima está descrito um trecho de código da parte de comandos, mais especificamente o comado se. Como é intuitivo, o arquivo possui um switch
	para filtra todos os comando correntes. Entrando no caso que lhe pertença, é criado um nó MasterNode e é desempilhado exatamente três posições
	na pilha, que se tornarão os "filhos" do comando se, e logo após este comando se empilha. Isso é feito análogamente para todos os outros
	comandos e expressões do programa, sendo o número de "filhos"/desempilhamentos, de acordo com a gramática construída.
	\\
	
	Dois casos especiais são os tratamentos de variáveis e funções. Para isso é aproveitado a estrutura utilizada no trabalho 3. Quando é
	encontrada uma variável, o ponteiro de Variável que está dentro da estrutura MasterNode é apontado para essa variável encontrada. E no caso
	de funções, isso será discutido na próxima seção.
	
\section{Árvores de execução e Árvore de execução de Função}
	Tudo que foi descrito nas seções anteriores, fazem parte da construção de ambas as árvores. Aqui será discutido algumas peculiaridades da
	contrução. Primeiramente, a árvore de execução nada mais é que a \textit main() do programa e a árvore de execução de função são as 
	funções contidas no programa. Ambas funcionam da mesma maneira, porém quando iniciado a função principal do programa é que a pilha esteja 
	vazia, para ocorrer os procedimentos já citados. Sendo assim, a partir da pilha criada para as instruções da função, é criado uma lista
	de MasterNode que representa a àrvore de execução de função, que é armazenada, em um ponteiro para MasterNode, que está dentro da estrutura
	de Função. 
	\\
	Após avaliadas as funções, como as declarações devem ser realizadas obrigatóriamente antes da main, todas as instruções a seguir pertencerão a
	árvore de execução do programa. Sendo assim, com o conteúdo que ainda está na pilha, é construída a árvore de execução.
	
\section{Executando a Árvore de Execução}	
	Para executar a árvore de execução, seja ela de função, ou não, foi criado o arquivo \textit executa.c; Nele possui a função 
	\textit executaArvore, que recebe como parâmetro um MasterNode. Nesta função é avaliada todas as expressões e comandos contidas na
	árvore, utilizando recursão para avaliação de todas as folhas da mesma.
	\\
	A função \textit executaComando trata todos os comandos e expressões escritos no documento. Para cada uma dela existe uma ação
	préviamente definido, e com isso a linguagem pode ser interpretada em cima da linguagem C.
	
\section{Considerações finais}		
	Certamente este trabalho é mais difícil que o três anteriores juntos, e talvez seja o trabalho mais difícil do curso. As primeira dificuldade
	foi a compreensão do problema, que foi resolvida após uma conversa em sala de aula com a professora. Logo após, veio a dificuldade de montar
	a árvore de execução. A priore a dificuldade era ter uma boa estratégia para montagem da mesma, e tendo a ideia implementá-la. A implementação
	da estratégia do empilhamento foi demorada e cheia de bugs que foram sendo encontrados e consertados ao decorrer do desenvolvimento. Montada a
	árvore, mais dificuldades: como executar? Com certeza a ideia de usar a recursão é fundamental e foi implementada. Não conseguimos executar
	todos os comandos da linguagem, como por exemplo, chavear, mas quase todos estão funcionando e podem ser observados nos exemplos.
	\\
	Por fim, a implementalçao do interpretador como um todo, ou seja, todos os analisadores anteriores, é fundamental para a compreensão de como 
	funciona um compilador. A implementação envolve praticamente todas as TAD's estudadas no curso e muita programação. Consideramos um trabalho
	pesado e talvez o mais difícil do curso. Tudo isso serve para demostrar o quão difícil é o desenvolvimento de um compilador para uma linguagem
	de programação.
	
	
	
	
	
	
	
	
	
	
	
	
	
	
	
	
	
	
	
	
	
	


\end{document}
